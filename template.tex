%%% template.annotated.tex
%%%
%%% This LaTeX source document can be used as the basis for your technical
%%% paper or abstract. Unlike ``template.tex,'' this version of the source
%%% document contains documentation of each of the commands and definitions
%%% that should be used in the preparation of your formatted document.
%%% 
%%% The parameter given to the ``acmsiggraph'' LaTeX class in the 
%%% ``\documentclass'' command controls several features of the formatted 
%%% output: the presence or absence of hyperlinked icons just prior to the 
%%% first section of the paper, the amount of space left clear for the ACM
%%% copyright notice, the presence or absence of line numbers and submission
%%% ID, and the presence or absence of an appropriate ``preprint'' notice.
%%% 
%%% If you are preparing a paper for presentation in the Technical Papers
%%% program at one of our two annual flagship conferences, held in North 
%%% America (SIGGRAPH) or Asia (SIGGRAPH Asia), you should use ``tog''
%%% as the parameter.
%%%
%%% If you are preparing a paper for presentation at one of our sponsored
%%% events, including SIGGRAPH and SIGGRAPH Asia, but not in those events' 
%%% Technical Papers program, or a one- to four-page abstract, you should 
%%% use ``conference'' as the parameter.
%%% (Technical Briefs and Game Papers presented at our annual flagship 
%%% events fall into this category, as do papers accepted to other SIGGRAPH-
%%% sponsored events, such as I3D or ETRA or VRCAI, as do the one-page 
%%% abstracts which serve as the primary documentation for many of our 
%%% annual conference programs, including Posters, Talks, and Emerging 
%%% Technologies.)
%%%
%%% If you are preparing a version of your content for review, you should
%%% use ``review'' as the parameter. Line numbers will be added to your 
%%% paper, and the submission ID value will be printed across the top of 
%%% each page of your paper. (Use the submission ID as the parameter to the
%%% ``TOGonlineID'' command, below.)
%%%
%%% If you are preparing a preprint of your content, you should use
%%% ``preprint'' as the parameter. This is primarily for annual conference
%%% papers; a header reading ``To appear in ACM TOG X(Y)'' will appear on
%%% each page of the formatted output (where X is the volume and Y is the 
%%% number of the issue in which it will be published).

\documentclass[tog]{acmsiggraph}

\usepackage{siunitx}

%%% Definitions and commands that begin with ``\TOG'' are meant to be used
%%% in the preparation of papers to be presented in the Technical Papers
%%% program at one of our annual flagship events - SIGGRAPH and SIGGRAPH 
%%% Asia. You can safely ignore these definitions and commands if your 
%%% content is to be presented in some other venue.

%%% ``\TOGonlineid'' should be filled with the online ID value you received
%%% when you submitted your technical paper. It will be printed out if you 
%%% prepare a ``review'' version of your paper.

\TOGonlineid{45678}

%%% Should your technical paper be accepted, you will be given three pieces
%%% of information: the volume and number of the issue of the ACM Transactions
%%% on Graphics journal in which your paper will be published, and the 
%%% ``article DOI'' value, which is unique to your paper and provides the 
%%% link to your paper's page in the ACM Digital Library. Fill in the 
%%% ``\TOGvolume,'' ``\TOGnumber,'' and ``\TOGarticleDOI'' definitions with
%%% the three pieces of information you receive.

\TOGvolume{0}
\TOGnumber{0}
\TOGarticleDOI{1111111.2222222}

%%% By default, your technical paper will contain hyperlinked icons which 
%%% point to your paper's article page in the ACM Digital Library, and to 
%%% the paper itself in the ACM Digital Library. You may wish to add one 
%%% or more links to your own resources. If any of the following four 
%%% definitions have URLs in them, an appropriate hyperlinked icon will be
%%% added to the list. 

\TOGprojectURL{}
\TOGvideoURL{}
\TOGdataURL{}
\TOGcodeURL{}

%%% Define the title of your paper here. Use capital letters as appropriate.
%%% Setting the entire title in upper-case letters is not correct, nor is 
%%% capitalizing only the first letter of the title.

\title{Simulating Magnets}

%%% Define the author list in the ``\author'' command. The ``\thanks'' 
%%% field can be used to define an e-mail address for the author.
%%% The ``\pdfauthor'' field should contain a comma-separated list of the
%%% authors of the paper, and is used, along with the title and keyword
%%% data, for PDF metadata. (To see this metadata, open the PDF in Adobe 
%%% Reader and select ``File > Properties > Description.''

\author{Richard DeVries\thanks{e-mail:rdevries@student.cs.uwaterloo.ca}\\University of Waterloo}
\pdfauthor{Richard DeVries}

%%% User-defined keywords.

\keywords{magnets, magnetic induction}

%%% End of the document preamble, start of the document.

\begin{document}

%%% A ``teaser'' image appears below the title and affiliation and above
%%% the two-column body of the paper. This is optional, but if you wish
%%% to include such an image, the commented-out code, below, can be used
%%% as an example. Please note that the inclusion of a ``teaser'' image
%%% may move the copyright space to the bottom of the right-hand column
%%% on the first page of your formatted output. This is acceptable.

%% \teaser{
%%   \includegraphics[height=1.5in]{images/sampleteaser}
%%   \caption{Spring Training 2009, Peoria, AZ.}
%% }

%%% The ``\maketitle'' command uses the author and title information 
%%% defined above, and prepares the formatted title.

\maketitle

%%% The ``abstract'' environment should contain the abstract for your
%%% content -- one to several paragraphs which describe the work.

\begin{abstract}

This report describes the physics-based animation concepts required to transform a simple particle simulator into a somewhat less simple magnet simulator. In particular, there are four concepts required to perform this transformation. First, since magnets attract each other, they will inevitably come into contact. This requires collision detection and handling. Second, magnets rotate and move about during the simulation. This requires implementing the rigid body equations of motion. Third, magnetic forces and torques need to be calculated so that the motion of the magnets can be determined. To do this, the magnetic induction caused by the magnets at arbitrary points must be calculated. The magnetic induction can also be used to draw the familiar magnetic field lines. Finally, the magnetic forces and torques must be applied to the magnets.

%Citations can be done this way~\cite{Jobs95} or this more concise 
%way~\shortcite{Jobs95}, depending upon the application.

\end{abstract}

%%% The ``CRCatlist'' environment defines one or more ACM ``Computing Review''
%%% (or ``CR'') categories, used for indexing your work. For more information
%%% on CR categories, please see http://www.acm.org/class/1998.

\begin{CRcatlist}
  \CRcat{I.3.5}{Computer Graphics}{Computational Geometry and Object Modelling}{Physically Based Modelling};
\end{CRcatlist}

%%% The ``\keywordlist'' prints out the user-defined keywords.

\keywordlist

%%% If you are preparing a paper to be presented in the Technical Papers
%%% program at one of our annual flagship events (and, therefore, using 
%%% the ``tog'' parameter to the ``\documentclass'' command), the 
%%% ``\TOGlinkslist'' command prints out the list of hyperlinked icons.
%%% If you are using any other parameter to the ``\documentclass'' command
%%% this command does absolutely nothing.

%\TOGlinkslist

%%% The ``\copyrightspace'' command will leave clear an amount of space
%%% at the bottom of the left-hand column on the first page of your paper,
%%% according to the parameter used in the ``\documentclass'' command.

\copyrightspace

%%% The first section of your paper. 

\section{Introduction}

Magnets and magnetic objects are frequently encountered in real life. Fridge magnets are used to attach shopping lists to the fridge, magnets are used in various electronics like computer hard drives, and there are children's toys the make use of metal balls and plastic rods with magnets on the ends. Despite how common magnets are, however, there are not many papers in the computer graphics community that explore magnetism. This report aims to give an overview of the basic concepts required to create a magnet simulator.

As this report presents merely the basics of a magnet simulator, not everything about simulating magnets and magnetism will be covered. Objects with permanent magnetization (that is, objects are magnetic on their own) are covered in this report, whereas objects with induced magnetization (that is, objects that become magnetic when exposed to a magnetic field) are not. In terms of the children's toy with the metal balls and plastic rods with magnets, the magnets in the plastic rods have permanent magnetization, whereas the metal balls have induced magnetization. This can be seen by the interaction between two of the metal balls. By themselves, two of the metal balls will not attract each other; however, when one of the metal balls is attached to the end of one of the plastic rods, the metal balls will stick to each other because of the induced magnetization in the balls. If the reader is interested in learning about objects with induced magnetization, they are encouraged to read ~\cite{Thomaszewski:2008:MIM}.

!!!The remainder of this report proceeds as follows. Section 2 briefly describes the work that this report is based on. Section 3 will describe the rigid body animation concepts used in this project. Section 4 will describe the magnetism concepts used in this project. Section 5 will present and discuss some results produced by this project. Finally, section 6 will discuss some shortcomings of the project and some future work that would improve it.

\section{Previous Work}

Rigid body simulation is a much-explored area of physics-based animation. However, for this project, only some basic parts of rigid body animation are required, so this project uses ~\cite{pixarnotes} for the required background on rigid body animation. These course notes give an excellent introduction to many aspects of physically based modelling, including differential equations, particle dynamics, rigid body dynamics, collision detection and handling, and a few others. The sections on rigid body dynamics and collision detection and handling are the most important sections of the course notes for this project.

As mentioned above, there has not been a lot of research into animating magnets in the computer graphics community. One paper that does explore magnetism is ~\cite{Thomaszewski:2008:MIM}. This paper covers the basics of magnets, including calculating magnetic induction and the forces and torques applied by the magnetic induction. This is the part of the paper that is implemented in this project. That paper also covers induced magnetism, which is not implemented in this project, as mentioned above. One further interesting topic explored by that paper is superconductors and their interaction with magnets.

\section{Rigid Body Dynamics}

We now go into detail about the rigid body dynamics concepts required for this project. As this project is focused on the basics, the description of these concepts will be primarily focused on one of the most basic shapes, the sphere. Some high-level explanation of how to extend these ideas to more general shapes will also be given.

\subsection{Collision Detection and Handling}

The first concept from rigid body dynamics that is explained is collision detection and handling. This is important for a magnet simulator because, in many circumstances, magnets will come into contact at some point in the simulation. If the collisions are not handled, the simulation will be highly physically inaccurate and, as will be seen later, the magnetic forces calculated by the simulator can be extremely high, causing the system to ``blow up.''

Before the positions and velocities of objects can be adjusted to account for collisions, the collisions must be detected. For spheres, detecting collisions is quite simple: if the distance between the centres of two spheres is less than the sum of their radii, then the spheres are intersecting, so they have collided.

For other shapes, collision detection is more complicated than it is for spheres, because other shapes are not as uniform. For general convex polyhedra, one method of testing for intersections is separating planes. As explained in ~\cite{pixarnotes}, if two convex polyhedra are non-intersecting, ``then a separating plane exists with the following property: either the plane contains a face of one of the polyhedra, or the plane contains an edge from one of the polyhedra and is parallel to an edge of the other polyhedra.'' Although finding these separating planes can take quite a bit of time, they can frequently be reused, since they can be associated with a face or an edge of a polyhedra. For more information on general collision detection, see Section 7 of the rigid body dynamics chapter of ~\cite{pixarnotes}.

One method of handling the collisions is using penalty forces. This method allows objects to inter-penetrate, but when they do, a force is applied to the two objects to push them apart. This force is proportional to the distance the objects have inter-penetrated, so that the more the objects overlap, the stronger the force is that pushes them apart. That is, the force applied on object A for overlapping object B is:
\begin{equation}
 F = kd\hat{n}
\end{equation}
where $k$ is a constant that can be adjusted to produce softer or harder collisions, $d$ is the distance that object A has penetrated into object B, and $\hat{n}$ is the unit vector normal to the surface of B. For two colliding spheres, $d$ is the sum of the radii of the spheres minus the distance between the centres of the two spheres, and $\hat{n}$ is the unit vector pointing from the centre of one sphere to the centre of the other. The direction of $\hat{n}$ is chosen based on which sphere's penalty force is being calculated.

\subsection{Rigid Body Equations of Motion}

The second concept from rigid body dynamics that is crucial for a magnet simulator is the set of equations describing rigid body motion. These equations allow the simulated magnets to rotate so that opposing poles can align and to move towards and away from each other as the magnetic forces dictate.

It is useful to first look at what information is required to advance a rigid body in a simulator. As rigid bodies, do not change their shape as the simulation progresses, their motion can be considered a combination of translations of and rotations about the centre of mass of the object. To update these throughout a simulation, it is necessary to store some information about the current state of the translations and rotations, as well as some information about how the translations and rotations are changing. This can be done by storing the following four pieces of information: the location of the centre of mass $x(t)$ of the rigid body, a rotation matrix $R(t)$ defining the orientation of the rigid body (quaternions are generally preferred for this, but for simplicity, we stick with a rotation matrix), the linear momentum $P(t)$ of the rigid body, and the angular momentum $L(t)$ of the rigid body.

To update these values from one time step to the next, the derivatives of these values are required. Though it may possibly seem strange that linear and angular momentums are stored rather than linear and angular velocities, the required derivatives are quite simple to calculate this way. By the definition of linear momentum:
\begin{equation}
 v(t) = \frac{P(t)}{m}
\end{equation}
where $m$ is the mass of the rigid body. This gives the derivative of $x(t)$. Additionally, the derivative of $P(t)$ is simply $F(t)$, the sum of the forces applied to the object, and the derivative of $L(t)$ is simply $\tau(t)$, the sum of the torques applied to the object.

Finding the derivative of the rotation matrix is a bit more complicated. First, the inertia tensor $I(t)$ is required. As described in ~\cite{pixarnotes}, this is ``the scaling factor between angular momentum [...] and angular velocity''. This inertia tensor is dependent on the rotation of the object, though it can be computed easily as
\begin{equation}
 I(t) = R(t)I_{body}R(t)^{T}
\end{equation}
where $R(t)$ is the rotation of the object and $I_{body}$ is the inertia tensor of the unrotated object. For a solid sphere,
\begin{equation}
 I_{body} =
   \begin{bmatrix}
   	\frac{2}{5}mr^2 & 0 & 0\\
   	0 & \frac{2}{5}mr^2 & 0\\
   	0 & 0 & \frac{2}{5}mr^2\\
   \end{bmatrix}
\end{equation}
where $m$ is the mass and $r$ is the radius of the sphere.

Now that the inertia tensor $I(t)$ has been found, the angular velocity $\omega(t)$ can be found:
\begin{equation}
\omega(t) = I(t)^{-1}L(t)
\end{equation}
Since finding matrix inverses can be costly, this can be implemented easier by noting that, for rotation matrices,
\begin{equation}
R(t)^{T} = R(t)^{-1}
\end{equation}
which gives:
\begin{equation}
I(t)^{-1} = (R(t)I_{body}R(t)^{T})^{-1} = R(t)I_body^{-1}R(t)^{T}
\end{equation}
Since $I_body$ is constant for a rigid body, so is $I_body^{-1}$, so the inverse only needs to be calculated once.

Finally, the derivative of the rotation matrix can be found, which is $\omega(t)^{\star}R(t)$.

Having found the derivatives of the four values stored for each rigid body, any time integration scheme can be used to advance the rigid bodies through the simulation.

For additional detail on how these values are calculated, as well as a discussion on using quaternions rather than rotation matrices, see sections 2, 3, and 4 of the Rigid Body Dynamics chapter of ~\cite{pixarnotes}.

\section{Magnet Simulation}

We now describe the basic concepts required to simulate magnets.

\subsection{Calculating Magnetic Induction}

In order to simulate magnets, the forces and torques applied by the magnets need to be calculated. However, computing the forces and torques relies on finding the magnetic induction, so this is covered first.

We first consider the case of a small magnetic dipole. Let $r_O$ be the location of the dipole, and let the vector $m$ be the magnetization of the dipole. The length of $m$ defines the strength of the magnetic dipole, and the direction of $m$ is the direction in which the north pole of the dipole points. Then the magnetic induction $B(r)$ caused by this magnetic dipole at a point $r$ is:
\begin{equation}
B(r) = \frac{\mu_0}{4\pi}\left[\frac{3n(n\cdot m) - m}{\left|r - r_O\right|^3}\right]
\end{equation}
where $n$ is a unit vector pointing in the same direction as $(r - r_O)$. $\mu_0$ is the magnetic constant \SI[per-mode=fraction]{4\pi e-7}{\volt \second \per \ampere \per \metre}.

This equation can be used to approximate the magnetic induction caused by magnets with volume. Large magnets can be divided up into many smaller components, with each component's magnetic properties being treated as a dipole at its centre.

Given a set of magnets, the total magnetic induction at a point can be found by simply summing the magnetic inductions caused by each component of each magnet. Since the total magnetic induction can be found at any point in the scene, and since this magnetic induction is a vector, any iterative method can be used to draw a series of line segments approximating the magnetic induction lines. This produces the familiar loops often shown in diagrams of magnets. In the examples below, starting points were chosen on the magnets, and an explicit forward Euler iterative scheme was used to advance the iteration.

\subsection{Calculating Magnetic Forces and Torques}

As explained in ~\cite{Thomaszewski:2008:MIM}, the force $F$ and torque $T$ applied by a magnetic induction $B$ on a dipole with magnetization $m$ are approximated by:
\begin{equation}
F = \nabla(m \cdot B)
\end{equation}
\begin{equation}
T = m \times B
\end{equation}

These two equations can be combined with the calculation of $B(r)$ by summing (8) over all components of all magnets that are not the magnet that the forces and torques are being calculated for. This gives the following equations for calculating the forces and torques applied to a component $k$ of a magnet:
\begin{equation}
\begin{split}
F_k = \frac{\mu_0}{4\pi}\sum_{i=1}^N \frac{1}{\left|r_k - r_i\right|^4}\left[-15n_{ik}\bigl((m_k \cdot n_{ik})(m_i \cdot n_{ik})\bigr) \right. \\
\left. + 3n_{ik}(m_k \cdot m_i) + 3\bigl(m_k(m_i \cdot n_{ik}) + m_i(m_k \cdot n_{ik})\bigr)\right]
\end{split}
\end{equation}
\begin{equation}
T_k = \frac{\mu_0}{4\pi}\sum_{i=1}^N \left[ \frac{3(m_k \times n_{ik})(m_i \cdot n_{ik}) - m_k \times m_i}{\left| r_k - r_i \right|^3} \right]
\end{equation}
Note that subscripts of $k$ indicate values corresponding to the component $k$. Also note that the summation is over the components of the magnets that do not contain the component $k$. As in previous equations, $r_j$ is the location of the component $j$, and $m_j$ is the magnetization of the component $j$. $n_{ik}$ is the unit vector pointing from the location of the component $i$ to the location of the component $k$.

One interesting thing to note about these equations is the exponent on the $\frac{1}{\left|r_k - r_i\right|^4}$ factor. For the force equation, the exponent is $4$, while in the torque equation, the exponent is $3$. Thus, if two magnets are far apart, the torque will be much larger than the force. This causes the magnets to rotate, aligning opposite poles, more than move towards each other. This can be seen in the example animations.

For further information about calculating magnetic induction, forces, and torques, an interested reader is encouraged to see ~\cite{Thomaszewski:2008:MIM}.

\section{Results}

The results shown in the included video are now discussed. In all of the animations, the wireframe cube has a side length of 1 metre. The spheres have a radius of 5 centimetres, a mass of 1 kilogram, and a magnetic moment of 300 joules per tesla.

The first animation clip is of three single component, spherical magnets in a line. These magnets are aligned so that they are all attracted towards each other. Since the magnets are evenly spaced out, the centre magnet stays stationary, as expected. When the magnets come into contact, the penalty forces push them apart again. After the second contact, the magnets bounce farther apart than they did after the first contact. The reason for this is discussed in the next section.

<Discuss other animations>

\section{Exposition}

\begin{equation}
 \sum_{j=1}^{z} j = \frac{z(z+1)}{2}
\end{equation}

\begin{eqnarray}
x & \ll & y_{1} + \cdots + y_{n} \\
  & \leq & z
\end{eqnarray}

\section{Exposition}

Lorem ipsum dolor sit amet, consectetur adipisicing elit, sed do
eiusmod tempor incididunt ut labore et dolore magna aliqua. Ut enim ad
minim veniam, quis nostrud exercitation ullamco laboris nisi ut
aliquip ex ea commodo consequat. Duis aute irure dolor in
reprehenderit in voluptate velit esse cillum dolore eu fugiat nulla
pariatur. Excepteur sint occaecat cupidatat non proident, sunt in
culpa qui officia deserunt mollit anim id est laborum.
\begin{figure}[ht]
  \centering
  \includegraphics[width=1.5in]{images/samplefigure}
  \caption{Sample illustration.}
\end{figure}
Lorem ipsum dolor sit amet, consectetur adipisicing elit, sed do
eiusmod tempor incididunt ut labore et dolore magna aliqua. Ut enim ad
minim veniam, quis nostrud exercitation ullamco laboris nisi ut
aliquip ex ea commodo consequat. Duis aute irure dolor in
reprehenderit in voluptate velit esse cillum dolore eu fugiat nulla
pariatur. Excepteur sint occaecat cupidatat non proident, sunt in
culpa qui officia deserunt mollit anim id est laborum.

\section{Conclusion}

Lorem ipsum dolor sit amet, consectetur adipisicing elit, sed do
eiusmod tempor incididunt ut labore et dolore magna aliqua. Ut enim ad
minim veniam, quis nostrud exercitation ullamco laboris nisi ut
aliquip ex ea commodo consequat. Duis aute irure dolor in
reprehenderit in voluptate velit esse cillum dolore eu fugiat nulla
pariatur. Excepteur sint occaecat cupidatat non proident, sunt in
culpa qui officia deserunt mollit anim id est laborum.

\section*{Acknowledgements}

The author would like to thank Christopher Batty for teaching the CS888 course this term and for the particle system starter code.

%%% Please use the ``acmsiggraph'' BibTeX style to properly format your
%%% bibliography.

\bibliographystyle{acmsiggraph}
\bibliography{template}
\end{document}
